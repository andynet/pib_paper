\chapter*{Introduction}
\addcontentsline{toc}{chapter}{Introduction}
\markboth{Introduction}{Introduction}

Cancer is a second leading cause of death globally, with an estimated 1 out of 6 deaths to be caused by cancer. \cite{cancer}
It is a large collection of diseases that can affect any tissue in the body.
Cancer cells are abnormal cells that grow beyond their usual boundaries. 
All types of these cells start to reproduce uncontrollably.
These uncontrolled cells may form masses of tissue, which are called tumours. \cite{cancer2}
In later stages, these tumours can spread into surrounding tissue and form metastases.
Metastases are a significant cause of death from cancer.

A correct cancer diagnosis is essential for efficient treatment of cancer because every type of cancer requires a specific treatment regimen consisting of one or more procedures such as radiotherapy, chemotherapy and surgery.
In this work, we will build models for detecting some of the characteristic features of the disease.
Specially, we will design artificial neural networks for classification of tumorous tissue using gene expression data from RNA sequencing.

Artificial neural networks are systems inspired by the biological neural networks in animals brains.
They are highly versatile models able to capture non-linear relationships between many predictors.
The concept of neural network dates back at least to the 1960s. 
However, it was difficult to use them in practice. \cite{schmidhuber2015deep}
The difficulty was mainly caused by an insufficient amount of labelled data and by the low computing power of the technology.
Because of the growth of computing power and the discovery of backpropagation, a technique to train these models efficiently, neural networks grew on the popularity amongst researchers.
Nowadays, these models became widely used in many applications, and they proved themselves to be of high accuracy even for complex problems.

The goal of this project will be to use neural networks for classification.
First, we will build a model to predict the primary site of tumorous tissue.
The solution to this problem might be especially useful in metastatic cancer, where the clinicians can find tumorous cells all around the body of a patient.
It is of high importance to know the primary site of the tissue, to establish the most efficient treatment.

The second objective will be to identify the stage of cancer.
This result could tell us more about the severity of the illness and could also potentially help to choose better-suited cure.

We divided our work into three main parts.

Firstly, chapter Materials and Methods, where we will describe data available for the analysis, their corresponding formats and the source databases. 
In this chapter, we will also provide a detailed description of each step of our pipeline, and we will show what effects these steps had for our data.

In the next chapter, we will present the models build and we will show the results these models produced.
Here, we will also provide visualisations of our most important findings.

In the Discussion chapter, we will evaluate our methods.
We will interpret our findings, and we will discuss what could cause them.
We will also look at the advantages and disadvantages of selected methods, and we will propose possible improvements over them.
